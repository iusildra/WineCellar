\documentclass[english,12pt,twoside,a4paper]{report}
\usepackage[semibold]{sourcesanspro}
\usepackage[utf8]{inputenc}
\usepackage{mathpazo}
\usepackage{sectsty}
\usepackage[english]{babel}
\usepackage[T1]{fontenc}
\usepackage{fancyhdr}
\usepackage{xcolor} % où xcolor selon l'installation
\usepackage{scrextend} % Forcer la 4eme  de couverture en page pair
\usepackage{multirow} %% Pour mettre un texte sur plusieurs rangées
\usepackage[absolute]{textpos} 
\usepackage{graphicx, wrapfig}
\usepackage{geometry}
\usepackage{hyperref}
\usepackage{titlesec}
\usepackage{ulem, contour}
\usepackage{array}
\usepackage{tikz}
\usepackage{multicol, multicolrule}
\usepackage{caption, subcaption}
\usepackage{booktabs}
\usepackage{listings}
\usepackage{adjustbox}
\usepackage[indent=20pt]{parskip}
\usepackage{courier}
\usepackage{nameref}
\usepackage{lastpage}
\usepackage{pifont}

\definecolor{Valentia}{RGB}{233,78,82}
\definecolor{Titleblue}{RGB}{114, 146, 162}

\SetMCRule{extend-fill=false}

\graphicspath{{./img}, {../img}}

\pagestyle{fancy}
\fancyhf{}
\fancyheadoffset{0.005\textwidth}
\fancyhead[LE,RO]{\slshape \rightmark}
\fancyhead[LO,RE]{\slshape \leftmark}
\lfoot{My Chief Cook Domain Analysis}
\rfoot{Page \thepage\ of \pageref*{LastPage}}

\definecolor{black}{rgb}{0,0,0}
\definecolor{green}{rgb}{0,0.5,0}
\definecolor{red}{rgb}{1,0,0}
\definecolor{blue}{rgb}{0,0,1}
\hypersetup{
	colorlinks=true,
	breaklinks=true,
	linkcolor=black,
	urlcolor=cyan,
	pdftitle={My Chief Cook Domain Analysis}}

\definecolor{codegreen}{rgb}{0,0.6,0}
\definecolor{codegray}{rgb}{0.5,0.5,0.5}
\definecolor{codepurple}{rgb}{0.58,0,0.82}
\definecolor{backcolour}{rgb}{0.95,0.95,0.95}
\lstdefinestyle{myStyle}{
  backgroundcolor=\color{white},   % choose the background color
  basicstyle=\footnotesize,        % the size of the fonts that are used for the code
  breakatwhitespace=false,         % sets if automatic breaks should only happen at whitespace
  breaklines=true,                 % sets automatic line breaking
  captionpos=b,                    % sets the caption-position to bottom
  commentstyle=\color{codegreen},    % comment style
  deletekeywords={...},            % if you want to delete keywords from the given language
  escapeinside={\%*}{*)},          % if you want to add LaTeX within your code
  extendedchars=true,              % lets you use non-ASCII characters; for 8-bits encodings only, does not work with UTF-8
  firstnumber=1,                   % start line enumeration with line 1
  frame=tb,                        % adds a frame around the code
  framerule=0.25pt,                % frame thickness
  keepspaces=true,                 % keeps spaces in text, useful for keeping indentation of code (possibly needs columns=flexible)
  keywordstyle=\color{blue},       % keyword style
  language=java,                   % the language of the code
  gobble=2,                        % the number of blank spaces before code
  morekeywords={},                 % if you want to add more keywords to the set
  numbers=left,                    % where to put the line-numbers; possible values are (none, left, right)
  numbersep=5pt,                   % how far the line-numbers are from the code
  numberstyle=\tiny\color{black},  % the style that is used for the line-numbers
  rulecolor=\color{black},         % if not set, the frame-color may be changed on line-breaks within not-black text (e.g. comments (green here))
  showspaces=false,                % show spaces everywhere adding particular underscores; it overrides 'showstringspaces'
  showstringspaces=false,          % underline spaces within strings only
  showtabs=false,                  % show tabs within strings adding particular underscores
  stepnumber=1,                    % the step between two line-numbers. If it's 1, each line will be numbered
  stringstyle=\color{codepurple},  % string literal style
  tabsize=2,                       % sets default tabsize to 2 spaces
}

\lstset{style=myStyle}

\renewcommand{\ULdepth}{1.8pt}
\contourlength{0.8pt}

\newcommand{\ul}[1]{%
	\uline{\phantom{#1}}%
	\llap{\contour{white}{#1}}%
}

\renewcommand{\footrulewidth}{1pt}
\captionsetup{labelfont={it, bf}, textfont={it}}
\urlstyle{same}

\newcommand{\resizeW}[1]{\resizebox{\linewidth}{!}{#1}}

\newcommand{\image}[4]{
  \begin{figure}[ht]
    \centering
    \fbox{\includegraphics[width=#2\linewidth]{#1}}
    \caption{#3}
    \label{fig:#4}
  \end{figure}
}
%%%%%%%%%%%%%%%%%%%%%%%%%%%%%%%%%%%%%%%%%%%%%%%%%%%%%%%%%%%%%%%

% \titleformat{\section}
% {\titlerule
% \vspace{.8ex}%
% \Large\bfseries}
% {\thesection.}{.5em}{}

% \titleformat{\part}[display]
% {\bfseries\Large}
% {\filleft PARTIE \Huge\thepart}
% {0ex}
% {\titlerule
% \vspace{1ex}%
% \filright}
% [\vspace{1ex}%
% \titlerule]

\captionsetup{labelfont={it, bf}, textfont={it}}
\setlength{\headheight}{15pt}
\setcounter{tocdepth}{1}

\begin{document}

\begin{titlepage}

  \newgeometry{left=2.5cm, bottom=3cm, top=2cm, right=2.5cm}

  \tikz[remember picture,overlay] \node[opacity=0.2,inner sep=0pt] at (73.6mm, -105mm){\includegraphics[scale=1.5]{logo-2x.png}};

  {\fontfamily{phv}\fontseries{mc}\selectfont
  %*****************************************************
  %******************** TITRE **************************
  %*****************************************************
  \centering
  \color{Valentia}
  \fontsize{18}{13}\selectfont
  \textbf{My Chief Cook}

  \normalsize
  \color{black}

  \bigskip
  \textbf{Computer Science and Management}

  \bigskip
  \textbf{Polytech Montpellier}

  \bigskip

  \color{Titleblue}
  \fontsize{17}{20.4}\selectfont
  \vspace{4cm}
  \textbf{DOMAIN ANALYSIS}\\

  %*****************************************************

  \vspace{4cm}
  \fontsize{15}{18}\selectfont
  \color{black}
  \bigskip

  \vspace{2cm}
  \normalsize
  \textbf{Presented by}\\
  \bigskip
  \fontsize{10}{12}\selectfont
  \vspace{1.5mm}
  \begin{table}[h]
    \centering
    \begin{tabular}{p{8cm}r}
      \toprule
      \textbf{Corentin Clément}      & \textbf{[Lead, Analytics]}          \\
      \textbf{Alexis Fondard Martin} & \textbf{[Development, Tests]}       \\
      \textbf{Anais Velcker}         & \textbf{[Documentation, Use cases]} \\
      \textbf{Richard Martin}        & \textbf{[Graphical User Interface]} \\
      \textbf{Lucas Nouguier}        & \textbf{[Database, Reports]}                 \\
      \bottomrule
    \end{tabular}
  \end{table}

  %************************************
  %**  LOGO  UNIVERSITÉ
  %*****************************************************
  \vspace{\fill}
  \begin{center}
    \includegraphics[height=60px]{LogoPolytech.png}
    \hfill
    \includegraphics[height=65px]{LogoUM.png}
  \end{center}
  }
\end{titlepage}

%%%%%%%%%%%%%%%%%%%%%%%%%%%%%%%%%%%%%%%%%%%%%%%%%%%%%%%%%%%%%%
\newgeometry{top=2cm, bottom=2.5cm, left=2cm, right=2cm}
\setlength{\columnsep}{1cm}
\tableofcontents
\clearpage

\part{Domain Analysis}
\image{model.png}{0.96}{Domain analysis}{domainAnalysis}

\part{Management report}
\chapter{Meetings description}
\section{Meeting 1: 21/11/2022}
First group meeting for the domain analysis diagram. We did a brainstorming to find the different classes with the different links/connections from a diagram that Lucas made. Everyone participated, proposing changes to better define the diagram.
\section{Meeting 2: 23/11/2022}
Following the teacher's remarks, we made changes with object-oriented thinking. We all worked on the connections together to complete the diagram.

\chapter{Personal impressions}
Supervising this step was really easy because the people in the group are serious, they do the required/necessary work by themselves.

In my opinion, the time planned for the deposit was maybe a bit short or there was a missing session with the teacher at the beginning of the week.

Maybe a word about the fact that I am supposed to supervise the whole team in general. Not everybody works that hard in the team for each step, but I understand (there is motivation for the project, tastes as some prefer design, others development). Still, everybody participates anyway, so I would say that for the moment everything is going well.
\end{document}